\title{InclusionAndExclusion}


\home{https://timechess.github.io/InclusionAndExclusion}
\github{https://github.com/timechess/InclusionAndExclusion}
\dochome{https://timechess.github.io/InclusionAndExclusion/doc}

% \home{localhost:8000}
% \dochome{localhost:8000/doc}

\maketitle


\tableofcontents

\nocite{*} % Delete this line if you have citations.

\section{Introduction}


\section{FinInter}

\begin{definition}\label{List.FinInter}
  \leanok
  \lean{List.FinInter}
  Define the intersection of a list of finite sets indexed by \verb|β|, where each set is given by a function \verb|A : β → Finset α|.
\end{definition}

\begin{lemma}\label{List.eq_FinInter}
  \lean{List.eq_FinInter}
  \uses{List.FinInter}
  \leanok
  Prove that an element is in the intersection of a non-empty list of finite sets if and only if it is in each of the sets.
\end{lemma}

\begin{proof}
  \leanok
  If there's only one element in the list, then it is trival.

  Then we prove by induction.

  Assume the statement hold for a list whose length is $n$, thus an element is in the intersection is in every one of the $n$ sets.

  We add a set $S$ to the beginning of the list. For any element $x$ in the intersection of the $n+1$ sets (which is a subset of the intersection of the previous $n$ sets), it must belongs to the intersection of the previous $n$ sets and $S$. Therefor, $x\in$ the intersection of the $n+1$ sets.

  Thus, the statement holds.

\end{proof}

\begin{definition}\label{Multiset.FinInter}
  \lean{Multiset.FinInter}
  \uses{List.FinInter,List.eq_FinInter}
  \leanok
  Define the intersection of a multiset of finite sets indexed by \verb|β|, where each set is given by a function \verb|A : β → Finset α|.
\end{definition}

\begin{proof}
  \leanok
  To prove that if sets \( A \) and \( B \) are equal, then \( A \cup C = B \cup C \), we can approach the proof by showing two inclusions: \( A \cup C \subseteq B \cup C \) and \( B \cup C \subseteq A \cup C \).

  \textbf{1. Show \( A \cup C \subseteq B \cup C \):}

  \begin{itemize}
    \item Take any element \( x \in A \cup C \).
    \item By the definition of union, \( x \in A \cup C \) means that \( x \) is either in \( A \) or in \( C \).
    \item If \( x \in A \), and since \( A = B \), it follows that \( x \in B \).
    \item Therefore, \( x \in B \cup C \) because \( x \) is in \( B \) (or in \( C \) if it were in \( C \)).
    \item Hence, every element of \( A \cup C \) is also in \( B \cup C \), which proves \( A \cup C \subseteq B \cup C \).
  \end{itemize}

  \textbf{2. Show \( B \cup C \subseteq A \cup C \):}

  \begin{itemize}
    \item Take any element \( x \in B \cup C \).
    \item By the definition of union, \( x \in B \cup C \) means that \( x \) is either in \( B \) or in \( C \).
    \item If \( x \in B \), and since \( A = B \), it follows that \( x \in A \).
    \item Therefore, \( x \in A \cup C \) because \( x \) is in \( A \) (or in \( C \) if it were in \( C \)).
    \item Hence, every element of \( B \cup C \) is also in \( A \cup C \), which proves \( B \cup C \subseteq A \cup C \).
  \end{itemize}

  Since we have shown both inclusions, we conclude that \( A \cup C = B \cup C \).
\end{proof}

\begin{lemma}\label{Multiset.eq_FinInter}
  \lean{Multiset.eq_FinInter}
  \uses{List.FinInter,List.eq_FinInter,Multiset.FinInter}
  \leanok
  Prove that an element is in the intersection of a multiset of finite sets if and only if it is in each of the sets.
\end{lemma}

\begin{proof}
  \leanok
  (Waited to be written...)
\end{proof}

\begin{definition}\label{FinInter₀}
  \lean{FinInter₀}
  \uses{Multiset.FinInter}
  \leanok
  Define the intersection of all finite sets indexed by \verb|β|, where each set is given by a function \verb|A : β → Finset α|.
\end{definition}

\begin{lemma}\label{eq_FinInter₀}
  \lean{eq_FinInter₀}
  \uses{Multiset.eq_FinInter,FinInter₀}
  \leanok
  Prove that the intersection of all finite sets indexed by \verb|β| is equal to the intersection of their corresponding sets in \verb|Set α|.
\end{lemma}

\begin{proof}
  \leanok
  (Waited to be written...)
\end{proof}

\section{FinUnion}

\begin{definition}\label{List.FinUnion}
  \leanok
  \lean{List.FinUnion}
  Define the union of a list of finite sets indexed by \verb|β|, where each set is given by a function \verb|A : β → Finset α|.
\end{definition}

\begin{lemma}\label{List.eq_FinUnion}
  \lean{List.eq_FinUnion}
  \uses{List.FinUnion}
  Prove that an element is in the union of a list of finite sets if and only if it is in at least one of the sets.
\end{lemma}

\begin{definition}\label{Multiset.FinUnion}
  \lean{Multiset.FinUnion}
  \uses{List.FinUnion,List.eq_FinUnion}
  Define the union of a multiset of finite sets indexed by \verb|β|, where each set is given by a function \verb|A : β → Finset α|.
\end{definition}

\begin{lemma}\label{Multiset.eq_FinUnion}
  \lean{Multiset.eq_FinUnion}
  \uses{List.FinUnion,List.eq_FinUnion,Multiset.FinUnion}
  Prove that an element is in the union of a multiset of finite sets if and only if it is in at least one of the sets.
\end{lemma}

\begin{definition}\label{FinUnion₀}
  \lean{FinUnion₀}
  \uses{Multiset.FinUnion}
  Define the union of all finite sets indexed by \verb|β|, where each set is given by a function \verb|A : β → Finset α|.
\end{definition}

\begin{lemma}\label{eq_FinUnion₀}
  \lean{eq_FinUnion₀}
  \uses{Multiset.eq_FinUnion,FinUnion₀}
  Prove that the union of all finite sets indexed by \verb|β| is equal to the union of their corresponding sets in \verb|Set α|.
\end{lemma}

\section{ToInt}

\begin{definition}\label{toInt}
  \leanok
  \lean{toInt}
  Assign a value to a proposition. If the proposition holds, assign a value of 1; otherwise, assign a value of 0.
\end{definition}

\begin{lemma}\label{toInt_and}
  \lean{toInt_and}
  \uses{toInt}
  \leanok
  The value of both propositions P and Q holding is equal to the value of P times the value of Q.
\end{lemma}

\begin{lemma}\label{toInt_not}
  \lean{toInt_not}
  \uses{toInt}
  \leanok
  The value of the negation of a proposition P is equal to 1 minus the value of P.
\end{lemma}

\begin{definition}\label{char_fun}
  \leanok
  \lean{char_fun}
  \uses{toInt}
  Define a characteristic function that returns 1 if \verb|x| is in the finite set \verb|A|, and 0 otherwise.
\end{definition}

\begin{lemma}\label{char_fun_inter}
  \lean{char_fun_inter}
  \uses{toInt,toInt_and,char_fun}
  The characteristic function of the intersection of two finite sets \verb|A| and \verb|B| is equal to the product of their characteristic functions.
\end{lemma}

\begin{lemma}\label{char_fun_union}
  \lean{char_fun_union}
  \uses{toInt,toInt_and,toInt_not,char_fun}
  The characteristic function of the union of two finite sets \verb|A| and \verb|B| is equal to 1 minus the product of 1 minus their characteristic functions.
\end{lemma}
\section{Auxiliary}

\begin{definition}\label{Finset.powerset₀}
  \leanok
  \lean{Finset.powerset₀}
  Define the set of all nonempty subsets of a given finite set \verb|A|.
\end{definition}

\begin{lemma}\label{card_eq}
  \lean{card_eq}
  If a finite set \verb|A| is equal to a set \verb|B| in the sense of set theory, then they have the same number of elements.
\end{lemma}

\begin{lemma}\label{mul_expand₃}
  \lean{mul_expand₃}
  Formalize the polynomial expansion of the product \(\prod_{i=0}^{n-1} (1 - g(i))\).
\end{lemma}

\begin{lemma}\label{mul_expand₂}
  \lean{mul_expand₂}
  \uses{Finset.powerset₀,mul_expand₃}
  Formalize the polynomial expansion of \(1 - \prod_{i=0}^{n-1} (1 - g(i))\).
\end{lemma}

\begin{lemma}\label{mul_expand₁}
  \lean{mul_expand₁}
  \uses{Finset.powerset₀,mul_expand₂}
  Formalize the polynomial expansion of \(1 - \prod_{i=0}^{n-1} (1 - g(i))\) in the context of finite sets of natural numbers up to \verb|n|.
\end{lemma}


\section{MainTheorem}

\begin{lemma}\label{card_eq_sum_char_fun}
  \lean{card_eq_sum_char_fun}
  \uses{toInt,char_fun,card_eq}
  \leanok
  The cardinality of a subset \verb|B| of \verb|A| is equal to the sum of the characteristic function of \verb|B| over \verb|A|.
\end{lemma}

\begin{proof}
  \leanok
  The right-hand side of the equation is equivalent to the number of elements in the intersection of sets $A$ and $B$. The left-hand side of the equation represents the number of elements in set $B$. Since $B$ is a subset of $A$, the intersection of $A$ and $B$ is equal to $B$. Therefore, the equality holds.

  Mathematically, this can be stated as:
  \begin{equation*}
  |B| = |A \cap B|
  \end{equation*}
  given that $B \subseteq A$.
\end{proof}

\begin{lemma}\label{card_eq_FinInter}
  \lean{card_eq_FinInter}
  \uses{FinInter₀,eq_FinInter₀,card_eq}
  The cardinality of the intersection of a family of finite sets indexed by \verb|β| is equal to the cardinality of their intersection in \verb|Finset| form.
\end{lemma}

\begin{lemma}\label{card_eq_FinUnion}
  \lean{card_eq_FinUnion}
  \uses{card_eq,FinUnion₀,eq_FinUnion₀}
  The cardinality of the union of a family of finite sets indexed by \verb|β| is equal to the cardinality of their union in \verb|Finset| form.
\end{lemma}

\begin{lemma}\label{char_fun_FinInter}
  \lean{char_fun_FinInter}
  \uses{FinInter₀,eq_FinInter₀,toInt,char_fun}
  \leanok
  The characteristic function of the intersection of a family of finite sets indexed by \verb|β| is equal to the product of their characteristic functions.
\end{lemma}

\begin{proof}
  \leanok
  Let $\mathcal{F} = \{A_1, A_2, \ldots, A_n\}$ be a collection of sets and let $B = \bigcap_{A \in \mathcal{F}} A$ be the intersection of these sets. Consider the characteristic function $\chi_B(x)$ for any element $x$. We wish to establish the following equivalence:

  \begin{equation*}

  \chi_B(x) = \prod_{A \in \mathcal{F}} \chi_A(x)
  \end{equation*}

  We analyze two cases:

  \subsection*{Case 1: $x$ belongs to the intersection $B$}

  \paragraph{Left Hand Side (LHS):} $\chi_B(x) = 1$. This condition is equivalent to $x$ belonging to every set $A \in \mathcal{F}$.

  \paragraph{Right Hand Side (RHS):} The product becomes $\prod_{A \in \mathcal{F}} \chi_A(x) = \prod_{A \in \mathcal{F}} 1 = 1$. Therefore, LHS = RHS.

  \subsection*{Case 2: $x$ does not belong to the intersection $B$}

  \paragraph{Left Hand Side (LHS):} $\chi_B(x) = 0$. This is equivalent to the existence of a set $A \in \mathcal{F}$ such that $x \not\in A$.

  \paragraph{Right Hand Side (RHS):} There exists an $A \in \mathcal{F}$ for which $\chi_A(x) = 0$. Since the product involves a zero, $\prod_{A \in \mathcal{F}} \chi_A(x) = 0$. Thus, LHS = RHS.

  In conclusion, for any element $x$, we have $\chi_B(x) = \prod_{A \in \mathcal{F}} \chi_A(x)$, which completes the proof.

\end{proof}

\begin{lemma}\label{char_fun_FinUnion}
  \lean{char_fun_FinUnion}
  \uses{FinUnion₀,eq_FinUnion₀,toInt,char_fun}
  The characteristic function of the union of a family of finite sets indexed by \verb|β| is equal to 1 minus the product of 1 minus their characteristic functions.
\end{lemma}

\begin{lemma}\label{mul_expand₀}
  \lean{mul_expand₀}
  \uses{Finset.powerset₀,mul_expand₁}
  The polynomial expansion of \(1 - \prod_{i=0}^{n-1} (1 - g(i))\).
\end{lemma}

\begin{theorem}\label{Principle_of_Inclusion_Exclusion}
  \lean{Principle_of_Inclusion_Exclusion}
  \uses{FinInter₀,FinUnion₀,eq_FinInter₀,eq_FinUnion₀,Finset.powerset₀,char_fun,card_eq_sum_char_fun,char_fun_FinInter,char_fun_FinUnion,card_eq,card_eq_FinInter,card_eq_FinUnion,mul_expand₀}
  The principle of inclusion-exclusion for a finite union of finite sets.
\end{theorem}

