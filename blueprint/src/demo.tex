
\begin{definition}\label{List.FinInter}
        \leanok
        \lean{List.FinInter}
                given finite number of finite sets, List.FinInter returns their intersection using an inductive way
    \end{definition}

\begin{definition}\label{List.FinUnion}
        \leanok
        \lean{List.FinUnion}
                given finite number of finite sets, List.FinInter returns their union using an inductive way
    \end{definition}

\begin{lemma}\label{List.eq_FinInter}
        \leanok
        \lean{List.eq_FinInter}
        \uses{List.FinInter}
                forall x : α, show that x ∈ (List.FinInter A L) ↔ forall i in L, x ∈ (A i)
    \end{lemma}

\begin{proof}
    \leanok
\end{proof}

\begin{lemma}\label{List.eq_FinUnion}
        \leanok
        \lean{List.eq_FinUnion}
        \uses{List.FinUnion}
                forall x : α, show that x ∈ (List.FinUnion A L) ↔ there exists an i in L, such that x ∈ (A i)
    \end{lemma}

\begin{proof}
    \leanok
\end{proof}

\begin{definition}\label{Multiset.FinInter}
        \leanok
        \lean{Multiset.FinInter}
        \uses{List.FinInter,List.eq_FinInter}
                we say the intersection of several finite sets does not depend on the order in which who and who intersect first. Therefore, we introduce the definition of multiset. Given a function A, we define a new function from a multiset to the intersection of finite sets whose index is in the multiset
    \end{definition}

\begin{definition}\label{Multiset.FinUnion}
        \leanok
        \lean{Multiset.FinUnion}
        \uses{List.FinUnion,List.eq_FinUnion}
                Same as above, we define a new function from a multiset to the union of finite sets whose index is in the multiset
    \end{definition}

\begin{lemma}\label{Multiset.eq_FinInter}
        \leanok
        \lean{Multiset.eq_FinInter}
        \uses{List.FinInter,List.eq_FinInter,Multiset.FinInter}
                We prove the lemma 'List.eq_FinInter' to be still true in the multiset case
    \end{lemma}

\begin{proof}
    \leanok
\end{proof}

\begin{lemma}\label{Multiset.eq_FinUnion}
        \leanok
        \lean{Multiset.eq_FinUnion}
        \uses{List.FinUnion,List.eq_FinUnion,Multiset.FinUnion}
                We prove the lemma 'List.eq_FinUnion' to be still true in the multiset case
    \end{lemma}

\begin{proof}
    \leanok
\end{proof}

\begin{definition}\label{FinInter₀}
        \leanok
        \lean{FinInter₀}
        \uses{Multiset.FinInter}
                Given a finite index set (@Finset.univ β _), we define FinInter₀ to be the intersection of all finite sets whose index's type is β
    \end{definition}

\begin{definition}\label{FinUnion₀}
        \leanok
        \lean{FinUnion₀}
        \uses{Multiset.FinUnion}
                Given a finite index set (@Finset.univ β _), we define FinUnion₀ to be the union of all finite sets whose index's type is β
    \end{definition}

\begin{lemma}\label{eq_FinInter₀}
        \leanok
        \lean{eq_FinInter₀}
        \uses{Multiset.eq_FinInter,FinInter₀}
                Same as above, we prove the lemma 'List.eq_FinInter' to be still true in the whole case
    \end{lemma}

\begin{proof}
    \leanok
\end{proof}

\begin{lemma}\label{eq_FinUnion₀}
        \leanok
        \lean{eq_FinUnion₀}
        \uses{Multiset.eq_FinUnion,FinUnion₀}
                Same as above, we prove the lemma 'List.eq_FinUnion' to be still true in the whole case
    \end{lemma}

\begin{proof}
    \leanok
\end{proof}

\begin{definition}\label{Finset.powerset₀}
        \leanok
        \lean{Finset.powerset₀}
                We define all the nonempty subsets of A to be A.powerset₀
    \end{definition}

\begin{definition}\label{toInt}
        \leanok
        \lean{toInt}
                We assign a value to a proposition. If the proposition holds, we assign a value of 1; otherwise, we assign a value of 0
    \end{definition}

\begin{lemma}\label{toInt_and}
        \leanok
        \lean{toInt_and}
        \uses{toInt}
                The value of P and Q both holds is equal to the value of P times the value of Q
    \end{lemma}

\begin{proof}
    \leanok
\end{proof}

\begin{lemma}\label{toInt_not}
        \leanok
        \lean{toInt_not}
        \uses{toInt}
                The value of ¬ P is equal to one sub the value of P
    \end{lemma}

\begin{proof}
    \leanok
\end{proof}

\begin{definition}\label{char_fun}
        \leanok
        \lean{char_fun}
        \uses{toInt}
                We define a function that if x ∈ A then returns 1, else returns 0
    \end{definition}

\begin{lemma}\label{card_eq_sum_char_fun}
        \leanok
        \lean{card_eq_sum_char_fun}
        \uses{toInt,char_fun,card_eq}
                Here we introduce a way to calculate the number of elements in B which is a subset of A
    \end{lemma}

\begin{proof}
    \leanok
\end{proof}

\begin{lemma}\label{char_fun_inter}
        \leanok
        \lean{char_fun_inter}
        \uses{toInt,toInt_and,char_fun}
                We claim that x ∈ (A ∩ B) is equal to x ∈ A and x ∈ B both holds
    \end{lemma}

\begin{proof}
    \leanok
\end{proof}

\begin{lemma}\label{char_fun_union}
        \leanok
        \lean{char_fun_union}
        \uses{toInt,toInt_and,toInt_not,char_fun}
                We claim that x ∈ (A ∪ B) is equal to at least one of x ∈ A and x ∈ B holds
    \end{lemma}

\begin{proof}
    \leanok
\end{proof}

\begin{lemma}\label{char_fun_FinInter}
        \leanok
        \lean{char_fun_FinInter}
        \uses{FinInter₀,eq_FinInter₀,toInt,char_fun}
                We claim that x ∈ (∩ i (A i)) is equal to forall i, x ∈ (A i) holds
    \end{lemma}

\begin{proof}
    \leanok
\end{proof}

\begin{lemma}\label{char_fun_FinUnion}
        \leanok
        \lean{char_fun_FinUnion}
        \uses{FinUnion₀,eq_FinUnion₀,toInt,char_fun}
                We claim that x ∈ (∪ i (A i)) is equal to at least one of (i : β, x ∈ (A i)) holds
    \end{lemma}

\begin{proof}
    \leanok
\end{proof}

\begin{lemma}\label{card_eq}
        \leanok
        \lean{card_eq}
                It's obvious that if a Finset A equals to a Set B in the view of Set, then they have the same number of elements. Since we use this lemma a lot in below proof, we put it here to be an independent lemma
    \end{lemma}

\begin{proof}
    \leanok
\end{proof}

\begin{lemma}\label{card_eq_FinInter}
        \leanok
        \lean{card_eq_FinInter}
        \uses{FinInter₀,eq_FinInter₀,card_eq}
                We derive it from eq_FinInter₀ and card_eq
    \end{lemma}

\begin{proof}
    \leanok
\end{proof}

\begin{lemma}\label{card_eq_FinUnion}
        \leanok
        \lean{card_eq_FinUnion}
        \uses{FinUnion₀,eq_FinUnion₀,card_eq}
                We derive it from eq_FinUnion₀ and card_eq
    \end{lemma}

\begin{proof}
    \leanok
\end{proof}

\begin{lemma}\label{mul_expand₃}
        \leanok
        \lean{mul_expand₃}
                Here we formalize the polynomial expansion of (∏ i (1 - g i))
    \end{lemma}

\begin{proof}
    \leanok
\end{proof}

\begin{lemma}\label{mul_expand₂}
        \leanok
        \lean{mul_expand₂}
        \uses{Finset.powerset₀,mul_expand₃}
                Same as above, here we formalize the polynomial expansion of (1 - ∏ i (1 - g i))
    \end{lemma}

\begin{proof}
    \leanok
\end{proof}

\begin{lemma}\label{mul_expand₁}
        \leanok
        \lean{mul_expand₁}
        \uses{Finset.powerset₀,mul_expand₂}
                Here we formalize the polynomial expansion of (1 - ∏ i (1 - g i)) in the view of (Finset (Fin n))
    \end{lemma}

\begin{proof}
    \leanok
\end{proof}

\begin{lemma}\label{mul_expand₀}
        \leanok
        \lean{mul_expand₀}
        \uses{Finset.powerset₀,mul_expand₁}
                Here we formalize the polynomial expansion of (1 - ∏ i (1 - g i)) in the view of (fun (Fin n) ↦ ℕ)
    \end{lemma}

\begin{proof}
    \leanok
\end{proof}

\begin{theorem}\label{Principle_of_Inclusion_Exclusion}
        \leanok
        \lean{Principle_of_Inclusion_Exclusion}
        \uses{FinInter₀,FinUnion₀,eq_FinInter₀,eq_FinUnion₀,Finset.powerset₀,char_fun,card_eq_sum_char_fun,char_fun_FinInter,char_fun_FinUnion,card_eq,card_eq_FinInter,card_eq_FinUnion,mul_expand₀}
                Finally, we can start to formalize the main theorem
    \end{theorem}

\begin{proof}
    \leanok
\end{proof}

